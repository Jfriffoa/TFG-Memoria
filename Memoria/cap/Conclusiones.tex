\section{Conclusiones}

%\textbf{LAS CONCLUSIONES SON GLOBALES DE TODO EL PROYECTO NO DE LOS RESULTADOS. DEBES HACERLO EN PLAN}

El objetivo de este proyecto era evaluar experimentalmente el impacto que los videojuegos y sus dilemas morales pueden tener sobre las personas. Para poder alcanzar este objetivo se han propuesto:

\begin{itemize}
    \item Un videojuego en el cual la toma de decisiones juega un papel fundamental. Se le hizo al jugador elegir que una chica viviera o muriese, y dependiendo de su respuesta el juego trata de influenciarlo para que al final del mismo cuando se le vuelva a hacer la misma pregunta al jugador, el jugador conteste la decisión contraria.
    \item Un cuestionario que el jugador debe resolver antes y después de jugar. Este cuestionario es el \gls{moral-foundation} y está hecho para medir la moral de los jugadores. Se utiliza el mismo cuestionario antes y después de jugar para ver si hay alguna variación en los resultados.
\end{itemize}
   
48 usuarios han jugado al videojuego y han respondido los cuestionarios pre- y post-. Los resultados obtenidos se han analizado y se ha comprobado que el juego implementado fue capaz de influenciar a los jugadores y hacer que su moral variase. No obstante, aquel cambio de moral no fue suficiente para que los jugadores cambiaran de decisión, ya que la gran mayoría mantuvo su opción inicial al terminar de jugar.

La elección del dilema moral fue el adecuado (\textit{el suicidio}) debido a que es una temática con la que el usuario objetivo lograba empatizar, sin embargo, lo corto del juego y el poco desarrollo de la narrativa pueden haber afectado a que no fuesen muchos usuarios los que cambiaron su decisión. Un videojuego más largo donde la narrativa se pudiese desarrollar y profundizar de mejor manera, agregando más elementos a la historia y dando una mejor oportunidad a que el jugador empatice más profundamente con ella, podría generar resultados más contundentes y convincentes.

La investigación logró cumplir los objetivos mencionados al inicio de este documento, el videojuego y las encuestas fueron diseñados e implementados con éxito, y hemos comprobado que -en mayor o menor medida- los videojuegos \textbf{si} influyen en nuestras decisiones.